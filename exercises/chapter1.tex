\section{Fields}

% Problem 4
\setcounter{problem}{3}
\begin{problem}
  The example of $\Z_p$ (where $p$ is prime) shows that not quite all laws of elementary arithmetic hold in fields;
  in $\Z_2$, for instance $1 + 1 = 0$.
  Prove that if $\mathcal{F}$ is a field, then either the results repeatedly adding 1 to itself is always different from 0, or else the first time that it is equal to 0 occurs when the number of summands is a prime.
\end{problem}

\begin{solution}
  Let $m \in \N$ be the lowest number of summands such that $\underbrace{1 + 1 + 1 + \ldots + 1}_{\text{m times}} = 0$.
  If $m$ were not prime, i.e., $m = ab$, then clearly $\underbrace{1 + 1 + 1 + \ldots + 1}_{\text{a times}} = x$, where $x \neq 0$ by the minimality of $m$.
  It is clear that $\underbrace{x + x + x + \ldots + x}_{\text{b times}} = \underbrace{1 + 1 + 1 + \ldots + 1}_{\text{m times}} = 0$, hence $x(\underbrace{1 + 1 + 1 + \ldots + 1}_{\text{b times}}) = 0$.
  Thus either $x = 0$ or $\underbrace{1 + 1 + 1 + \ldots + 1}_{\text{b times}} = 0$, both leading to a contradiction of the minimality of $m$.
\end{solution}
