\section{Dimension and Isomorphism}

% Problem 4.1
\begin{problem}
  \begin{enumerate}[label=(\alph*)]
    \item What is the dimension of the set $\C$ of all complex numbers considered as a real vector space?
    \item Every complex vector space $\mathcal{V}$ is intimately associated with a real vector space $\mathcal{V}^-$; 
      the space $\mathcal{V}^-$ is obtained from $\mathcal{V}$ by refusing to multiply vectors of $\mathcal{V}$ by anything other than real scalars.
      If the dimension of the complex vector space $\mathcal{V}$ is $n$, what is the dimension of the real vector space $\mathcal{V}^-$?
  \end{enumerate}
\end{problem}

\begin{solution}
  \begin{enumerate}[label=(\alph*)]
    \item Since every $z \in \C$ can be written as $\alpha \cdot 1 + \beta \cdot i$, it is clear that $\{1, i\}$ spans $\C$ over $\R$.
      These vectors are also linearly independent since $\alpha \cdot 1 = i$ implies $\alpha = i$, thus $\alpha \notin \R$.
      Hence $\{1, i\}$ is a basis of $\C$ over $\R$ and the vector space is 2-dimensional.
    \item Let $\mathcal{V}$ be a complex vector space isomorphic to $\C^n$, with the canonical basis.
      If we consider the set of vectors of $\C^n$ over $\R$ we see that with the previous basis we can only make vectors of $\R^n$.
      Thus we need to add a vector $(0, 0, \ldots, i, 0, \ldots, 0)$ for every vector in the canonical basis.
      Thus we end up with a set of $2n$ elements.
      It spans $\C^n$ over $\R$ and is linearly independent for the same reasons as above.
      Hence $\mathcal{V}^-$ is 2n-dimensional.
  \end{enumerate}
\end{solution}

% Problem 4.2
\begin{problem}
  Is the set $\R$ of all real numbers a finite-dimensional vector space over the field $\Q$ of all rational numbers?
\end{problem}

\begin{solution}
  No.
  Imagine we had a finite basis of $\R$ over $\Q$.
  Then we would have the following finite union of countable sets.
  Let $x_1, \ldots x_n$ be a basis, consider $x_i \cdot q$, for each $q \in \Q$.
  Since there are only countably many rationals, this forms a finite union of countable sets.
  We know that this union is itself countable.
  However this union is precisely equal $\R$, contradicting the uncoutability of $\R$, proven by Cantor.
\end{solution}

% Problem 4.3
\begin{problem}
  How many vectors are there in an n-dimensional vector space over the field $\Z_p$ where $p$ is prime?
\end{problem}

\begin{solution}
  Every vector could be uniquely represented by $\alpha_1 x_1 + \alpha_2 x_2 + \ldots \alpha_n x_n$ for some basis $\{x_1, \ldots, x_n\}$, with $\alpha_i \in \Z_p$.
  We have $p$ possibilities for each $\alpha_i$.
  Hence we can form $p^n$ vectors.
  Since the basis spans the vector sapce, there are precisely $p^n$ vectors.
\end{solution}

% Problem 4.4
\begin{problem}
  Discuss the following assertion: if two rational vector spaces have the same cardinal number, then they are isomorphic.
\end{problem}

\begin{solution}
  The given statement is false.
  Consider the sets $\Q$ and $\Q[\sqrt{2}]$ over $\Q$.
  It is easy to show that they are both countable, i.e., have a cardinal number $\aleph_0$.
  However, $\Q$ forms a 1-dimensional vector space with a basis $\{1\}$, while $\Q[\sqrt{2}]$ gives us a 2-dimensional vector space with a basis $\{1, \sqrt{2}\}$.
  Thus we have 2 vector spaces with the same cardinal number that are not isomorphic.
\end{solution}
