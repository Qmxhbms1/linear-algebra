\section{Dual basis, Reflexivity, Annihilators}

% Problem 7.1
\begin{problem}
  Define a non-zero linear functional $y$ on $\C^3$ such that if $x_1 = (1, 1, 1)$ and $x_2 = (1, 1, -1)$, then $[x_1, y] = [x_2, y] = 0$.
\end{problem}

\begin{solution}
  Since a linear functional is uniquely determined by where it sends basis vectors, we can define $y$ by putting $[x_1, y] = [x_2, y] = 0$ and taking $x_3 = (0, 1, 0)$ and putting $[x_3, y] = 1$.
  Clearly $y$ is non-zero.
\end{solution}

% Problem 7.4
\setcounter{problem}{3}
\begin{problem}
  If $y(x) = \xi_1 + \xi_2 + \xi_3$ whenever $x = (\xi_1, \xi_2, \xi_3)$ is a vector in $\C^3$, then $y$ is a linear functional on $\C^3$;
  find a basis of the subspace consisting of all those vectors $x$ for which $[x, y] = 0$.
\end{problem}

\begin{solution}
  Take the vectors $x_1 = (1, 0, -1)$ and $x_2 = (0, 1, -1)$.
\end{solution}

% Problem 7.7
\setcounter{problem}{6}
\begin{problem}
  If $V$ is an $n$-dimensional vector space over a finite field, and if $0 \le m \le n$ then the number of $n$-dimensional subspaces of $V$ is the same as the number of of $(n - m)$-dimensional subspaces.
\end{problem}

\begin{solution}
  Let $k$ be the number of elements of the finite field $F$.
  Then we know that $V$ has precisely $k^n$ vectors as it is isomorphic to $F^n$.
  Since the dual space of $V$ is also $n$-dimensional we know that it also has $k^n$ elements.
  Take any $m$-dimensional subspace $W$ of $V$.
  We know that its annihilator $W^0$ has dimension $(n - m)$.
  This creates a one-to-one correspondence between the $m$-dimensional subspaces of $V$ and the $(n - m)$-dimensional subspaces of $V'$.
  Since these two vector spaces have the same number of elements it follows that the amount of $m$ and $(n - m)$-dimensional subspaces is the same.
\end{solution}
