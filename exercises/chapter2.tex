\section{Vector Spaces}


% Problem 5
\setcounter{problem}{4}
\begin{problem}
  Consider the vector space $\mathcal{P}$ and the subsets $\mathcal{V}$ of $\mathcal{P}$ consisting of those vectors (polynomials) $x$ for which
  \begin{enumerate}[label=(\alph*)]
    \item $x$ has degree $3$,
    \item $2x(0) = x(1)$,
    \item $x(t) \ge 0$ whenever $0 \le t \le 1$,
    \item $x(t) = x(1 - t)$ for all $t$.
  \end{enumerate}
  In which of these cases is $\mathcal{V}$ a vector space?
\end{problem}

\begin{solution}
  \begin{enumerate}[label=(\alph*)]
    \item Trivial
    \item Notice that $2a_1 = a_n + a_{n - 1} + \ldots + a_2 + a_1$.
      Using this it is trivial to check that $(\alpha \cdot x = \beta \cdot y)(1) = 2(\alpha \cdot a_1 = \beta \cdot b_1) = 2(\alpha \cdot x(0) + \beta \cdot y(0))$.
      Hence this is a vector space.
    \item Consider $x(t) = 1$.
      Clearly it is is part of $\mathcal{V}$, but it has no inverse under vector addition.
    \item With some simple algebra we can see that this is a vector space.
  \end{enumerate}
\end{solution}

