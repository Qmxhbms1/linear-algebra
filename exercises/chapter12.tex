\section{Subspaces, Calculus of Subspaces and Dimension of a subspace}

% Problem 1
\begin{problem}
  If $\mathcal{V}$ and $\mathcal{W}$ are finite-dimensional subspaces with the same dimension, and if $\mathcal{W} \subset \mathcal{V}$, then $\mathcal{W} = \mathcal{V}$.
\end{problem}

\begin{solution}
  Let us consider an $x \in \mathcal{V}$ such that $x \notin \mathcal{W}$.
  Consider a basis $\mathcal{B} = \{x_1, \ldots, x_n\}$ of $\mathcal{W}$.
  Since we know that $x$ cannot be written as a linear combinantion of vectors in $\mathcal{B}$, the set $\mathcal{B} \cup \{x\}$ must be linearly independent.
  However this is a set with $n + 1$ elements, contradicting the fact that any subset with $m > n$ elements of an n-dimensional vectors space is linearly dependent.
  Hence, we cannot have $x \notin \mathcal{W}$, thus $\mathcal{V} \subset \mathcal{W}$ and finally $\mathcal{V} = \mathcal{W}$.
\end{solution}

% Problem 2
\begin{problem}
  If $\mathcal{W}$ and $\mathcal{U}$ are subspaces of a vector space $\mathcal{V}$, and if every vector in $\mathcal{V}$ belongs either to $\mathcal{W}$ or $\mathcal{U}$ (or both), then either $\mathcal{W} = \mathcal{V}$ or $\mathcal{U} = \mathcal{V}$ (or both).
\end{problem}

\begin{solution}
  Suppose neither $\mathcal{W}$ nor $\mathcal{U}$ were equal to $\mathcal{V}$.
  Then there would be some some $x \in \mathcal{W}$ and $y \in \mathcal{U}$ such that $x \notin \mathcal{U}$ and $y \notin \mathcal{V}$.
  Clearly $x + y \in \mathcal{V}$ as it is a linear combinantion of two vectors in a vector space.
  However, notice that $x + y \notin \mathcal{W}$, otherwise we would have $(x + y) - x = y \in \mathcal{W}$.
  Completely analogously, we have $x + y \notin \mathcal{U}$, contradicting our assumption that every vector of $\mathcal{V}$ is in one of the subspaces.
  Thus at least one of $\mathcal{W}$ or $\mathcal{U}$ must be equal to $\mathcal{V}$
\end{solution}

% Problem 3
\begin{problem}
  If $x, y$, and $z$ are vectors such that $x + y + z = 0$, then $x$ and $y$ span the same subspace as $y$ and $z$.
\end{problem}

\begin{solution}
  Consider any vector $v$ in the subspace spanned by $x$ and $y$.
  By definition, we have $v = \alpha x + \beta y$.
  However from our assumption notice that we know $x = - y - z$.
  Hence we have $v = \alpha (- y - z) + \beta y = - \alpha z + (\beta - 1) y$.
  Thus $v$ is in the subspace spanned by $y$ and $z$.

  We can make the same argument to show that for any $w$ in the subspace spanned by $y$ and $z$, we have that $w$ belongs to the subspace spanned by $x$ and $y$.

  Thus these subspaces are equal.
\end{solution}

% Problem 4
\begin{problem}
  Suppose that $x$ and $y$ are vectors and $\mathcal{W}$ is a subspace in a vector space $\mathcal{V}$;
  let $\mathcal{U}$ be the subspace spanned by $\mathcal{W}$ and $x$, and let $\mathcal{U'}$ be the subspace spanned by $\mathcal{W}$ and $y$.
  Prove that if $y$ is in $\mathcal{U}$ but not in $\mathcal{W}$, then $x$ is in $\mathcal{U'}$.
\end{problem}

\begin{solution}
  Let $\mathcal{B} = \{x_1 , \ldots, x_n\}$ be a basis of the subspace $\mathcal{W}$.
  If we had $x \in \mathcal{W}$ then we would have $\mathcal{W} = \mathcal{U}$ which would contradict our assumption that $y$ is in one but not the other.
  Thus $x \notin \mathcal{W}$ and $\mathcal{B} \cup \{x\}$ is linearly independent and clearly a basis of the subspace spanned by $\mathcal{W}$ and $x$.
  Since $y \in \mathcal{U}$, we know that for some scalars $\alpha_i$ we have $y = \alpha_1 x_1 + \ldots \alpha_n x_n + \alpha x$, and we know that $\alpha \neq 0$, otherwise $y$ would be a linear combinantion of vectors in $\mathcal{B}$ and hence belong to $\mathcal{W}$.
  Finally, we have $x = \frac{1}{\alpha} y - \frac{\alpha_1}{\alpha} x_1 - \ldots - \frac{\alpha_n}{\alpha} x_n$, hence $x$ is a linear combinantion of vectors in $\mathcal{U'}$ and $x \in \mathcal{U'}$.
\end{solution}

% Problem 5
\begin{problem}
  Suppose that $L, W$ and $V$ are subspaces of a vector space.
  \begin{enumerate}[label=(\alph*)]
    \item Show that the equation
      \[L \cap (W + V) = (L \cap W) + (L \cap V)\]
      is not necessarily true.
    \item Prove that
      \[L \cap (W + (L \cap V)) = (L \cap W) + (L \cap V).\]
  \end{enumerate}
\end{problem}

\begin{solution}
  \begin{enumerate}[label=(\alph*)]
    \item Consider the following counterexample.
      Let $L, W$ and $V$ be complex vectors spaces formed by the vectors of the form $\begin{bmatrix} 0\\ y\\ \end{bmatrix}, \begin{bmatrix} x\\ 0\\ \end{bmatrix}$ and $\begin{bmatrix} x\\ x\\ \end{bmatrix}$, respectively.
      The we can see the vector space $(W + V)$ consists of vectors $\begin{bmatrix} x\\ y\\ \end{bmatrix}$ and clearly $L \cap (W + V) = \{\begin{bmatrix} 0\\ y\\ \end{bmatrix}\}$.
      On the other hand, we have $L \cap W = \{\begin{bmatrix} 0\\ 0\\ \end{bmatrix}\}$ and $L \cap V = \{\begin{bmatrix} 0\\ 0\\ \end{bmatrix}\}$, thus $(L \cap W) + (L \cap V) = \begin{bmatrix} 0\\ 0\\ \end{bmatrix}$.
      Thus the two sides are not equal.
    \item Let $x \in L \cap (W + (L \cap V))$ then we know $x \in L$ and $x \in W + (L \cap V)$.
      Thus $x$ can be written as $x = u + v$ for some $u \in W$ and $v \in L \cap V$.
      We then have $v \in L$ and $v \in V$.
      Notice that since $x \in L$ and $v \in L$, we must also have $u = x - v \in L$.
      Hence $u \in L \cap W$ and $x = u + v \in (L \cap W) + (L \cap V)$ and $L \cap (W + V) \subset (L \cap W) + (L \cap V)$.

      Conversly, take some $x \in (L \cap W) + (L \cap V)$.
      We have $x = u + v$ for some $u \in L \cap W$ and $v \in L \cap V$.
      From there we see that $u \in L$, $u \in W$, $v \in L$ and $v \in V$.
      Since we have both $u \in L$ and $v \in L$, then clearly $u + v = x \in L$
      Since we have $u \in W$ and $v \in V$ it is clear that $x = u + v \in W + V$.
      Thus $x \in L \cap (W + V)$ and we have $(L \cap W) + (L \cap V) \subset L \cap (W + V)$.
      Finally we get our desired equality $L \cap (W + V) = (L \cap W) + (L \cap V)$.
  \end{enumerate}
\end{solution}

% Problem 6
\begin{problem}
  \begin{enumerate}[label=(\alph*)]
    \item Can it happen that a non-trivial subspace of a vector space $V$ (i.e., a subspace different from both $0$ and $V$) has a unique complement?
    \item If $W$ and $m$-dimensional subspace in an $n$-dimensional vector space, then every complement of $W$ has dimension $n - m$.
  \end{enumerate}
\end{problem}

\begin{solution}
  \begin{enumerate}[label=(\alph*)]
    \item Let $W$ be a subspace of the vector space $V$ and $U$ it's complement.
      Fix some $v \in V$ such that $v \notin W$ and $v \notin U$. (if such a $v$ doesn't exist then $W$ and $U$ are trivial).
      By the definition of complement we know that there exist some $x \in W$ and $y \in U$ such that $x + y = v$.
      Notice that from there we have $v - x = y$ and that $- x \in W$.
      Create a basis $\mathcal{B}$ of $U$ which contains $y$ (every linearly independent set can be extended to a basis).
      Consider the subspace $U'$ which spans the linearly independent set $(\mathcal{B} \setminus \{y\}) \cup \{v\}$.
      Since $v \notin W$, it is still clear that $W \cap U' = 0$.
      Clearly all vectors in $V$ which weren't written as linear combinantions of $y$ and some other vector from $W$ are in $W + U'$.
      For any vector $z = w + y$ we have $z = w - x + v$, with $w - x \in W$ and $v \in U'$.
      Thus $U'$ is a complement of $W$.
    \item This directly follows from the following problem.
  \end{enumerate}
\end{solution}

% Problem 7
\begin{problem}
  \begin{enumerate}[label=(\alph*)]
    \item Show that if both $W$ and $U$ are three-dimensional subspaces of a five-dimensional vector space, then $W$ and $U$ are not disjoint.
    \item If $W$ and $U$ are finite-dimensional subspaces of a vector space, then
      \[dim W + dim U = dim (W + U) + dim (W \cap U).\]
  \end{enumerate}
\end{problem}

\begin{solution}
  Let $W$ and $U$ be n-dimensional and m-dimensional subspaces respectively.
  Let $\mathcal{B}$ be a basis for the subspace defined by their intersection.
  Now, consider the basis $\mathcal{B_1}$ and $\mathcal{B_2}$ of $W$ and $U$ respectively, which contain $\mathcal{B}$ (we extend $\mathcal{B}$ to these basis).
  Notice that $W + U$ is precisely the set which spans the linearly independent set $\mathcal{B_1} \cup \mathcal{B_2}$.
  By the definitions of our basis we know $|\mathcal{B_1} \cup \mathcal{B_2}| = |\mathcal{B_1}| + |\mathcal{B_2}| - |\mathcal{B}|$.
  Thus we have $dim (W + U) = dim W + dim U - dim (W \cap U)$.
  It follows that $dim W + dim U = dim (W + U) + dim (W \cap U)$.

  It is trivial that (a) follows directly from this theorem.
\end{solution}

% Problem 8
\begin{problem}
  A polynomial $x$ is called \textit{even} if $x (- t) = x (t)$ identically in $t$ and it is called \textit{odd} if $x (- t) = - x (t)$.
  \begin{enumerate}[label=(\alph*)]
    \item Both the class $W$ of even polynomials and the class $U$ of odd polynomials are subspaces of the space $P$ of all (complex) polynomials.
    \item Prove that $W$ and $U$ are each other's complements.
  \end{enumerate}
\end{problem}

\begin{solution}
  \begin{enumerate}[label=(\alph*)]
    \item Let $x (t)$ and $y (t)$ be even polynomials.
      Consider
      \[(\alpha x + \beta y)(t) = \alpha x (t) + \beta y (t) = \alpha x (- t) + \beta y (- t) = (\alpha x + \beta y)(- t).\]
      Thus the set of even polynomials is a subspace.
      Similarly, for odd polynomials we have
      \[(\alpha x + \beta y)(t) = \alpha x (t) + \beta y (t) = - \alpha x (- t) - \beta y (- t) = - (\alpha x (- t) + \beta y (- t)) = - (\alpha x + \beta y)(- t).\]
    \item If we had $x (t) = x (- t) = - x (t)$ then we clearly have $x (t) = 0$.
      Thus their intersection is trivial.
      Notice that if a polynomial has only even degrees it must be even and similarly if it has only odd degrees it must be odd.
      Trivially, any polynomial can be split into the sum of two polynomials, one with only it's even degrees and one with only the odd ones.
      Thus the direct sum of the space of all even polynomials and the space of all odd polynomials is precisely the set of all polynomials.
  \end{enumerate}
\end{solution}


